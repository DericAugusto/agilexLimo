L'évolution des véhicules et robots autonomes représente une révolution dans notre manière de concevoir le transport et la robotique. Ces technologies, en constante amélioration, promettent de transformer radicalement nos vies en offrant des solutions de mobilité plus sûres, plus efficaces et plus respectueuses de l'environnement. Le marché des véhicules autonomes pourrait atteindre 1,5 trillion de dollars d'ici 2030, illustrant l'énorme potentiel économique de cette technologie \cite{McKinsey2020}. 

Le développement de ces systèmes autonomes repose sur des avancées en intelligence artificielle, notamment dans des domaines tels que le traitement d'images et l'apprentissage en profondeur, qui permettent aux machines de comprendre et de naviguer dans leur environnement avec une précision croissante. L'utilisation de ROS (\textit{Robot Operating System}) est également cruciale dans ce contexte \cite{Quigley2009}. Par ailleurs, l'intégration de capteurs avancés et de systèmes d'assistance à la conduite renforce la sécurité et la fiabilité des véhicules autonomes \cite{Anderson2014}. 

L'implication de l'ENSEM dans le développement du robot Agilex Limo s'inscrit dans cette dynamique en proposant des solutions innovantes et adaptées à ces défis. En préparant les ingénieurs à ces enjeux, ce projet contribue à façonner l'avenir de la mobilité et de la robotique autonome, un domaine à la pointe de l'innovation technologique.

Le travail développé est disponible dans le dépôt GitHub utilisé par l'équipe sur \cite{agilexLimoGithub}.


