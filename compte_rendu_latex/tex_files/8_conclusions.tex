En conclusion, ce projet sur le véhicule autonome Agilex Limo a intégré avec succès diverses technologies et méthodes pour le contrôle et la supervision du robot. Malgré les progrès significatifs, certains défis demeurent, notamment dans l'implémentation du nœud de commande de trajectoires automatiques et la publication en temps réel des commandes. L'utilisation potentielle de capteurs intégrés pour améliorer la navigation et d'autres fonctionnalités n'a pas été pleinement exploitée. De plus, la possibilité d'appliquer les connaissances en IoT pour la gestion à distance des nœuds représente une voie d'amélioration future. Ce travail illustre l'importance de l'intégration multidisciplinaire et de la résolution de problèmes pour le développement de systèmes autonomes complexes.